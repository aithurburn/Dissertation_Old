% Options for packages loaded elsewhere
\PassOptionsToPackage{unicode}{hyperref}
\PassOptionsToPackage{hyphens}{url}
%
\documentclass[
]{article}
\usepackage{amsmath,amssymb}
\usepackage{lmodern}
\usepackage{iftex}
\ifPDFTeX
  \usepackage[T1]{fontenc}
  \usepackage[utf8]{inputenc}
  \usepackage{textcomp} % provide euro and other symbols
\else % if luatex or xetex
  \usepackage{unicode-math}
  \defaultfontfeatures{Scale=MatchLowercase}
  \defaultfontfeatures[\rmfamily]{Ligatures=TeX,Scale=1}
\fi
% Use upquote if available, for straight quotes in verbatim environments
\IfFileExists{upquote.sty}{\usepackage{upquote}}{}
\IfFileExists{microtype.sty}{% use microtype if available
  \usepackage[]{microtype}
  \UseMicrotypeSet[protrusion]{basicmath} % disable protrusion for tt fonts
}{}
\makeatletter
\@ifundefined{KOMAClassName}{% if non-KOMA class
  \IfFileExists{parskip.sty}{%
    \usepackage{parskip}
  }{% else
    \setlength{\parindent}{0pt}
    \setlength{\parskip}{6pt plus 2pt minus 1pt}}
}{% if KOMA class
  \KOMAoptions{parskip=half}}
\makeatother
\usepackage{xcolor}
\usepackage[margin=1in]{geometry}
\usepackage{graphicx}
\makeatletter
\def\maxwidth{\ifdim\Gin@nat@width>\linewidth\linewidth\else\Gin@nat@width\fi}
\def\maxheight{\ifdim\Gin@nat@height>\textheight\textheight\else\Gin@nat@height\fi}
\makeatother
% Scale images if necessary, so that they will not overflow the page
% margins by default, and it is still possible to overwrite the defaults
% using explicit options in \includegraphics[width, height, ...]{}
\setkeys{Gin}{width=\maxwidth,height=\maxheight,keepaspectratio}
% Set default figure placement to htbp
\makeatletter
\def\fps@figure{htbp}
\makeatother
\setlength{\emergencystretch}{3em} % prevent overfull lines
\providecommand{\tightlist}{%
  \setlength{\itemsep}{0pt}\setlength{\parskip}{0pt}}
\setcounter{secnumdepth}{-\maxdimen} % remove section numbering
\pagestyle{plain}
\setcounter{tocdepth}{5}
\linespread{1.2}
\usepackage{setspace}
\rhead{DoPL and DOSPERT}
\usepackage{fancyhdr}
\cfoot{\thepage}
\fancyheadoffset[L]{0pt}
\fancyhf{}
\fancyhead[RO,LE]{\small\thepage}
\renewcommand{\headrulewidth}{0pt}
\interfootnotelinepenalty=10000
\newcommand{\HRule}{\rule{\linewidth}{0.25mm}}
\let\cleardoublepage=\clearpage
\usepackage{atbegshi}% http://ctan.org/pkg/atbegshi
\AtBeginDocument{\AtBeginShipoutNext{\AtBeginShipoutDiscard}}
\doublespacing
\usepackage{booktabs}
\usepackage{longtable}
\usepackage{array}
\usepackage{multirow}
\usepackage{wrapfig}
\usepackage{float}
\usepackage{colortbl}
\usepackage{pdflscape}
\usepackage{tabu}
\usepackage{threeparttable}
\usepackage{threeparttablex}
\usepackage[normalem]{ulem}
\usepackage{makecell}
\usepackage{xcolor}
\ifLuaTeX
  \usepackage{selnolig}  % disable illegal ligatures
\fi
\IfFileExists{bookmark.sty}{\usepackage{bookmark}}{\usepackage{hyperref}}
\IfFileExists{xurl.sty}{\usepackage{xurl}}{} % add URL line breaks if available
\urlstyle{same} % disable monospaced font for URLs
\hypersetup{
  hidelinks,
  pdfcreator={LaTeX via pandoc}}

\author{}
\date{\vspace{-2.5em}}

\begin{document}

\hypertarget{chapter-1}{%
\section{Chapter 1:}\label{chapter-1}}

\hypertarget{introduction}{%
\subsection{Introduction}\label{introduction}}

Throughout political history, tyrants, and despots have influenced great
power over large swaths of land and communities. One common thread
amongst these individuals is how they wield their great power, often
through dominant tactics such as threats and political subversion.
Recent history has shown with individuals like Donald Trump, Kim
Jong-Un, and Rodrigo Duterte who display authoritarian traits often
wield their power through fear and threats of violence
{[}@bernstein2020; @bynion2018; @kirby2021{]}. How this power is wielded
is often different for each individual. Some individuals such as Duterte
and Bolsonaro wielded their power more dramatically than the likes of
Trump. Individuals wielding power need not be tyrants such as the
former. Individuals like Angela Merkel used her position and leadership
skills to be a world leader in most negotiations. While individuals more
well known for their status demonstrated their power through prestige
motives. To better understand how individuals such as world leaders or
opinion makers gain and wield their power over others. Research in this
field is often difficult to research yet strides have been made to
understand power, namely through research in moral judgment and
decision-making such as power orientation.

\hypertarget{dominance-prestige-and-leadership-orientation}{%
\subsection{Dominance, Prestige, and Leadership
orientation}\label{dominance-prestige-and-leadership-orientation}}

Research in power desire motives has focused on three subdomains:
dominance, leadership, and prestige {[}@suessenbach2019{]}. Each of
these three different power motives is explained as to different ways or
methods that individuals in power sought power or were bestowed upon
them. Often these dominant individuals will wield their power with force
and potentially cause risk to themselves to hold onto that power.

\hypertarget{dominance}{%
\subsubsection{\texorpdfstring{\emph{Dominance}}{Dominance}}\label{dominance}}

The dominance motive is one of the more researched methods and
well-depicted power motives. Individuals with a dominant orientation
display the more primal human behavior. These individuals will seek
power through direct methods such as asserting dominance, control over
resources, or physically assaulting someone {[}@johnson2012;
@winter1993{]}. Early research in dominance motives has shown that acts
of dominance ranging from asserting physical dominance over another to
physical displays of violence have been shown in many mammalian species,
including humans {[}@petersen2018; @rosenthal2012{]}.

Individuals high in dominance are often high in Machiavellianism, and
narcissism, and often are prone to risky behavior (discussion further in
the next section). Continued research has hinted at a possible tendency
for males to display these dominant seeking traits more than females
{[}@sidanius2000; @bareket2020{]}. When individuals high in dominance
assert themselves they are doing so to increase their sense of power
{[}@anderson2012; @bierstedt1950{]}. Asserting one's sense of dominance
over another can be a dangerous task. In the animal kingdom, it can
often lead to injury. While, humans asserting dominance can take a
multitude of actions such as leering behaviors, physical distance, or
other non-verbal methods to display dominance {[}@petersen2018;
@witkower2020{]}. Power from a dominant perspective is not always
bestowed upon someone. Often, high dominance individuals will take
control and hold onto it.

\hypertarget{prestige}{%
\subsubsection{\texorpdfstring{\emph{Prestige}}{Prestige}}\label{prestige}}

Contrary to the dominant motivation of using intimidation and aggression
to gain more power, a prestige motivation or prestige, in general, is
bestowed upon an individual from others in the community {[}@maner2016;
@suessenbach2019{]}. Different from dominance motivation, prestige
motivation is generally unique to the human species {[}@maner2016{]}.
Due in part to ancestral human groups being smaller hunter-gatherer
societies, individuals that displayed and used important behaviors
beneficial to the larger group were often valued and admired by the
group. Therein, the social group bestows the authority onto the
individual. Generally, this type of behavior can be passively achieved
by the prestigious individual. However, this does not remove the intent
of the actor in that they too can see prestige from the group, but the
method of achieving that social status greatly differs from that of
dominance-seeking individuals.

Apart from dominance-motivated individuals that continually have to
fight for their right to have power over others, individuals that seek
or were given power through a prestige motivation are not generally
challenged in the same sense as dominant individuals. Displaying
behaviors that the community would see as beneficial would endear them
to the community making the survival of the community as a whole better
{[}@maner2016{]}. Evolutionarily this would increase the viability of
the prestigious individual and their genes. Similar to the dominance
perspective, the prestige perspective overall increases the power and
future survivability of the individual. However, due to the natural
difference between prestige and dominance, dominance-seeking individuals
are challenged more often resulting in more danger to their position
{[}@johnson2012{]}.

\hypertarget{leadership}{%
\subsubsection{\texorpdfstring{\emph{Leadership}}{Leadership}}\label{leadership}}

With a shared goal a leader is someone that takes initiative and
attracts followers for that shared goal {[}@vanvugt2006{]}. Leadership
is an interesting aspect of behavior in that it is almost exclusive to
human interaction. Discussions by evolutionary psychologists point to
the formation of early human hunter-gatherer groups where the close
interconnectedness created a breeding ground for leadership roles. As
early humans began to evolve it would become advantageous for
individuals to work together for a common goal {[}@king2009{]}. Often,
individuals with more knowledge of a given problem would demonstrate
leadership and take charge or be given power. Multiple explanations of
the evolution of leadership exist such as coordination strategies, and
safety, along with evidence for growth in social intelligence in humans
{[}@vanvugt2006; @king2009{]}.

An interesting aspect of leadership motivation is the verification of
the qualities of the leader by the communities. Individuals that are
often put into leadership roles or take a leadership role often display
the necessary goals, qualities, and knowledge to accomplish the
shared/stated goal. However, this is not always the case, especially for
those charismatic leaders who could stay on as a leader longer than the
stated goal requires {[}@vugt2014{]}. Traditionally, leadership was
thought to be fluid in that those with the necessary knowledge at the
time would be judged and appointed as the leader. However, these
charismatic leaders use their charisma, uniqueness, nerve, and talent to
hold onto their status.

\hypertarget{risk}{%
\subsection{Risk}\label{risk}}

Every time people leave the relative safety of their home, every
decision they make they are taking some form of risk. Financial risk is
often discussed in the media usually concerning the stock market.
However, the risk is not just present in finances but also in social
interactions such as social risk, sexual risk, health, and safety risk,
recreational, and ethical risks {[}@weber2002; @shearer2005;
@breakwell2007; @kuhberger2009{]}. Each individual is different in their
likelihood and perception of participating in those risks. Some will be
more inclined to be more financially risky while others would risk their
health and safety.

Whether to engage in a risky situation is very complex depending on a
cost-benefit analysis {[}@johnson2015a{]}. Do the positives outweigh the
negatives? In practice, not all individuals will do a cost-benefit
analysis of a risky situation. Often, the timing of an event makes such
an analysis disadvantageous. The benefits are often relative to the
individual decision-maker. Differences emerge in the general likelihood
to engage in risky behavior such that males tend to be more likely to
engage in risky behaviors than their female counterparts
{[}@desiderato1995; @chen2021{]}. Women tended to avoid risky situations
except for social risks.

\hypertarget{experiment-one}{%
\subsection{Experiment One}\label{experiment-one}}

\hypertarget{method}{%
\subsection{Method}\label{method}}

\hypertarget{participants}{%
\subsubsection{Participants}\label{participants}}

Participants were a convenience sample of 95 (Mage = NA, SD = NA)
individuals from Prolific Academic crowdsourcing platform
(``www.prolific.co''). Requirements for participation were: (1) be 18
years of age or older and (2) and as part of Prolific Academics policy,
have a prolific rating of 90 or above. Participants received £4 which
amounts to £8 an hour as compensation for completion of the survey.
Table 1 demonstrates the demographic information for experiment one.

\hypertarget{demographic-questionnaire}{%
\subsubsection{Demographic
Questionnaire}\label{demographic-questionnaire}}

Prior to the psychometric scales, participants were asked to share their
demographic characteristics (e.g., age, gender, ethnicity, ethnic
origin, and educational attainment).

\hypertarget{dominance-prestige-and-leadership-orientation-1}{%
\subsubsection{Dominance, Prestige, and Leadership
Orientation}\label{dominance-prestige-and-leadership-orientation-1}}

The 18-item Dominance, Prestige, and Leadership scale {[}DoPL;
@suessenbach2019{]}, is used to measure dominance, prestige, and
leadership orientation. Each question corresponds to one of the three
domains. Each domain is scored across six unique items related to those
domains (e.g., ``I relish opportunities in which I can lead others'' for
leadership). These statements and goals are rated on a scale from 0
(Strongly disagree) to 5 (Strongly agree). Internal consistency
reliability per subscale with the current sample with \(\alpha\)'s
ranging from dominance = 0.84, prestige = 0.75, leadership = 0.85, and
UMS = 0.75.

\hypertarget{spitefulness-scale}{%
\subsubsection{Spitefulness Scale}\label{spitefulness-scale}}

The Spitefulness scale {[}@marcus2014{]} is a measure with seventeen
one-sentence vignettes to assess the spitefulness of participants. The
original spitefulness scale has 31-items. In the original Marcus and
colleagues' paper, fifteen were removed. For the present study, however,
4-items were removed because they did not meet the parameters for the
study i.e., needed to be dyadic, interpersonal spitefulness. To follow
this, we replaced the four that were removed and included three
reverse-scored items from the original thirty-one. Example questions
included, ``It might be worth risking my reputation in order to spread
gossip about someone I did not like,'' and ``Part of me enjoys seeing
the people I do not like to fail even if their failure hurts me in some
way''. Items are scored on a 5-point scale ranging from 1 (``Strongly
disagree'') to 5 (``Strongly agree''). Higher spitefulness scores
represent higher acceptance of spiteful attitudes. Internal consistency
reliability for the current sample is \(\alpha\) = 0.84.

\hypertarget{sexuality-self-esteem-subscale}{%
\subsubsection{Sexuality Self-Esteem
Subscale}\label{sexuality-self-esteem-subscale}}

The Sexuality Self-Esteem subscale (SSES; @snell1989) is a subset of the
Sexuality scale that measures the self-esteem of participants. The
10-items chosen reflected questions on the sexual esteem of participants
on a 5-point scale of +2 (Agree) and -2 (Disagree). For ease of online
use the scale was changed to 1 (``Disagree'') and 5 (``Agree''), data
analysis will follow the sexuality scale scoring procedure. Example
questions are, ``I am a good sexual partner,'' and ``I sometimes have
doubts about my sexual competence.'' Higher scores indicate a higher
acceptance of high self-esteem statements. Internal consistency
reliability for the current sample is \(\alpha\) = 0.95.

\hypertarget{sexual-jealousy-subscale}{%
\subsubsection{Sexual Jealousy
Subscale}\label{sexual-jealousy-subscale}}

The Sexual Jealousy subscale by @worley2014 are 3-items from the 12-item
Jealousy scale. The overall jealousy scale measures jealousy in
friendships ranging from sexual to companionship. The 3-items are ``I
would worry about my partner being sexually unfaithful to me.'', ``I
would suspect there is something going on sexually between my partner
and their friend.'', and ``I would suspect sexual attraction between my
partner and their friend.'' The items are scored on a 5-point scale
ranging from 1 (``Strongly disagree'') to 5 (``Strongly agree''). Higher
scores indicate a tendency to be more sexually jealous. Internal
consistency reliability for the current sample is \(\alpha\) = 0.72.

\hypertarget{sexual-relationship-power-scale}{%
\subsubsection{Sexual Relationship Power
Scale}\label{sexual-relationship-power-scale}}

The Sexual Relationship Power Scale (SRPS; @pulerwitz2000) is a 23-item
scale that measures the overall power distribution in a sexually active
relationship. The SRPS is split into the Relationship Control
Factor/Subscale (RCF) and the Decision-Making Dominance Factor/Subscale
(DMDF). The RCF measures the relationship between the partners on their
agreement with statements such as, ``If I asked my partner to use a
condom, he{[}they{]} would get violent.'', and ``I feel trapped or stuck
in our relationship.'' Items from the RCF are scored on a 4-point scale
ranging from 1 (``Strongly agree'') to 4 (``Strongly disagree''). Lower
scores indicate an imbalance in the relationship where the participant
indicates they believe they have less control in the relationship.
Internal consistency reliability for the current sample is \(\alpha\) =
0.87.

The DMDF measures the dominance level of sexual and social decisions in
the relationship. Example questions include, ``Who usually has more say
about whether you have sex?'', and ``Who usually has more say about when
you talk about serious things?'' Items on the DMDF are scored on a
3-item scale of 1 (``Your Partner''), 2 (``Both of You Equally''), and 3
(``You''). Higher scores indicate more dominance by the participant in
the relationship. Internal consistency reliability for the current
sample is \(\alpha\) = 0.64.

\hypertarget{scenario-realism-question}{%
\subsubsection{Scenario Realism
Question}\label{scenario-realism-question}}

Following Worley and Samp in their 2014 paper on using
vignettes/scenarios in psychological studies, a question asking the
participant how realistic or how much they can visualize the scenario
is. The 1-item question is ``This type of situation is realistic.'' The
item is scored on a 5-point scale with how much the participants agreed
with the above statement, 1 (``Strongly agree'') to 5 (``Strongly
disagree''). Higher scores indicate disagreement with the statement and
reflect the belief that the scenario is not realistic.

\hypertarget{spiteful-vignettes}{%
\subsubsection{Spiteful Vignettes}\label{spiteful-vignettes}}

After participants complete the above scales, they are presented with
10-hypothetical vignettes. Each vignette was written to reflect a dyadic
or triadic relationship with androgynous names to control for gender.
Five vignettes have a sexual component while five are sexually neutral.
An example vignette is,

\begin{quote}
``Casey and Cole have been dating for 6 years. A year ago, they both
moved into a new flat together just outside of the city. Casey had an
affair with Cole's best-friend. Casey had recently found out that they
had an STI that they had gotten from Cole's best-friend. Casey and Cole
had sex and later Cole found out they had an STI.''
\end{quote}

For each vignette, the participant is asked to rate each vignette on how
justified they believe the primary individual, Casey in the above, is
with their spiteful reaction. Scoring ranges from 1 (``Not justified at
all'') to 5 (``Being very justified''). Higher scores overall indicate
higher agreement with spiteful behaviors.

\hypertarget{procedure}{%
\subsection{Procedure}\label{procedure}}

Participants were recruited on Prolific Academic. Participants must be
18-years of age or older, restriction by study design and Prolific
Academic's user policy. The published study is titled, ``Moral Choice
and Behavior''. The study description follows the participant
information sheet including participant compensation. Participants were
asked to accept their participation in the study. Participants were then
automatically sent to the main survey (Qualtrics, Inc.).

Once participants accessed the main survey, they were presented with the
consent form for which to accept they responded by selecting ``Yes''.
Participants were then asked to provide demographic characteristics such
as gender, ethnicity, and educational attainment. Participants would
then complete in order, the spitefulness scale, the sexual relationship
power scale, the sexual jealousy subscale, and sexuality self-esteem
subscale. Next, participants were presented ten vignettes where they
were instructed to rate on the level of justification for the action
carried out in the vignette. After each vignette, participants would
rate the realism of the scenario. Upon completion of the survey (median
completion time 20 minutes SD = 10 Minutes 30 seconds), participants
were shown a debriefing message and shown the contact information of the
Primary Investigator (Andrew Ithurburn). Participants were then
compensated at £8/hr. via Prolific Academic.

\hypertarget{data-analysis}{%
\subsection{Data Analysis}\label{data-analysis}}

Demographic characteristics were analyzed using a one-way analysis for
continuous variables (age) and Chi-squares tests for categorical
variables (sex, ethnicity, ethnic origin, and educational attainment).
Means and standard deviations were calculated for the surveys along with
correlational analyses (e.g., spitefulness, SESS, SRPS, SJS).

Bayesian multilevel models were used to test differences between levels
of justifications of vignettes that are either sexually or non-sexually
vindictive in behavior.

\hypertarget{results-and-discussion}{%
\subsection{Results and Discussion}\label{results-and-discussion}}

Ninety-Five individuals participated in the present experiment. A
majority of the participants in experiment 1 identified as male
(\emph{n} = 31). Table 1 shows the demographic information for
experiment 1. Table 2 presents the results of a Bayesian correlational
matrix of all measures. As evidenced in the Bayesian correlational
matrix, most surveys positively correlated with one another.

\hypertarget{spitefulness}{%
\subsubsection{Spitefulness}\label{spitefulness}}

For this analysis we used the Bayesian parameter estimation using R and
brms {[}@rcoreteam2021; @burkner2018{]}. An annotated r script file,
including all necessary information is available at
\url{https://osf.io/jz6qb}. On average, individuals were not rated as
being more spiteful, (\emph{M} = 33.92, \emph{SD} = 9.32, Min-max =
{[}16 - 57{]}). Justification as a function of the four indices was
moderately explained by the model (\emph{R\textsuperscript{2}} = 0.54).
We conducted an exploratory Bayesian correlation analysis on the data,
where we investigated correlations between 8 of the indices (e.g.,
Spite, Dominance, Prestige, Leadership, Sexual Jealousy, Sexual
Self-Esteem, and Sexual Relationship Power Scale).

Selected notable non-null correlations were found between Spite and
Sexual Jealousy (95\% CI: {[}0.12, 0.25{]}), Spite and Dominance (95\%
CI: {[}0.45, 0.54{]}), and Sexual Relationship Power and Dominance (95\%
CI: {[}0, 0.13{]}). Table 2 contains a complete list of all Bayesian
correlations.

\hypertarget{limitations-and-future-directions}{%
\subsection{Limitations and Future
Directions}\label{limitations-and-future-directions}}

\hypertarget{experiment-2}{%
\subsection{Experiment 2}\label{experiment-2}}

\hypertarget{methods}{%
\subsection{Methods}\label{methods}}

Materials remain the same in terms of the (1) Demographic Questionnaire,
(2) Dominance, Prestige, and Leadership Questionnaire, and (3) DOSPERT
Questionnaire. However, we added the Brief-Pathological Narcissism
Inventory to assess possible interactions of dominance and narcissism in
risky decision-making. Materials and methods were approved by the
University of \#\# Participants

Following experiment 1, participants were a convenience sample of 111
individuals from Prolific Academic's crowdsourcing platform
(www.prolific.io). Prolific Academic is an online crowdsourcing service
that provides participants access to studies hosted on third-party
websites. Participants were required to be 18 years of age or older and
be able to read and understand English. Participants received £4.00,
which is above the current minimum wage pro-rata in the United Kingdom,
as compensation for completing the survey. The Psychology Research
Ethics Committee at the University of Edinburgh approved all study
procedures {[}ref: 212-2021/2{]}. The present study was pre-registered
along with a copy of anonymized data and a copy of the R code is
available at (\url{https://osf.io/s4j7y}).

\hypertarget{materials}{%
\subsection{Materials}\label{materials}}

\hypertarget{brief-pathological-narcissism-inventory}{%
\subsubsection{\texorpdfstring{\emph{Brief-Pathological Narcissism
Inventory}}{Brief-Pathological Narcissism Inventory}}\label{brief-pathological-narcissism-inventory}}

The 28 item Brief Pathological Narcissism Inventory (B-PNI; Schoenleber
et al., 2015) is a modified scale of the original 52-item Pathological
Narcissism Inventory (PNI; Pincus et al., 2009). Like the PNI the B-PNI
is a scale measuring individuals' pathological narcissism. Items in the
B-PNI retained all 7 pathological narcissism facets from the original
PNI (e.g., exploitativeness, self-sacrificing self-enhancement,
grandiose fantasy, contingent self-esteem, hiding the self, devaluing,
and entitlement rage). Each item is rated on a 5 point Likert scale
ranging from 1 (not at all like me) to 5 (very much like me). Example
items include ``I find it easy to manipulate people'' and ``I can read
people like a book.''

\hypertarget{procedure-1}{%
\subsection{Procedure}\label{procedure-1}}

Participants were recruited via a study landing page on Prolific's
website or via a direct e-mail to eligible participants
{[}@prolificacademic2018{]}. The study landing page included a brief
description of the study including any risks and benefits along with
expected compensation for successful completion. Participants accepted
participation in the experiment and were directed to the main survey on
pavlovia.org (an online JavaScript hosting website similar to Qualtrics)
where they were shown a brief message on study consent.

Once participants consented to participate in the experiment they
answered a series of demographic questions. Once completed, participants
completed the Dominance, Prestige, and Leadership Scale and the Domain
Specific Risk-taking scale. An additional survey was added (the novel
aspect of experiment 2) where participants, in addition to the two
previous surveys, were asked to complete the brief-pathological
narcissism inventory. The three scales were counterbalanced to account
for order effects. After completion of the main survey, participants
were shown a debriefing statement that briefly mentions the purpose of
the experiment along with the contact information of the main researcher
(AI). Participants were compensated £4.00 via Prolific Academic.

\hypertarget{data-analysis-1}{%
\subsection{Data analysis}\label{data-analysis-1}}

Demographic characteristics were analyzed using multiple regression for
continuous variables (age) and Chi-square tests for categorical
variables (gender, race, ethnicity, ethnic origin, and education). Means
and standard deviations were calculated for the relevant scales (i.e.,
DoPL and DOSPERT). All analyses were done using {[}@rcoreteam2021{]}
along with {[}@burkner2017{]} package.

The use of bayesian statistics has a multitude of benefits to
statistical analysis and research design. One important benefit is
through the use of prior data in future analyses. Termed as priors, is
the use of prior distributions for future analysis. This allows for the
separation of how the data might have been collected or what the
intention was. In essence, the data is the data without the
interpretation of the scientist.

All relevant analyses were conducted in a Bayesian framework using the
brms package {[}@burkner2018{]} along with the cmdstanr packages notes
{[}@gabry2021{]}. In addition to the aforementioned packages, we used
bayestestR, rstan, and papaja for analysis along with the creation of
this manuscript {[}@makowski2019; @aust2020;
@standevelopmentteam2020{]}.

\hypertarget{results}{%
\subsection{Results}\label{results}}

\hypertarget{preregistered-analyses}{%
\subsection{Preregistered Analyses}\label{preregistered-analyses}}

\hypertarget{demographic-and-dopl}{%
\subsubsection{\texorpdfstring{\emph{Demographic and
DoPL}}{Demographic and DoPL}}\label{demographic-and-dopl}}

\hypertarget{domain-specific-risk-taking}{%
\subsection{Domain-Specific
Risk-Taking}\label{domain-specific-risk-taking}}

\hypertarget{interactions}{%
\subsection{Interactions}\label{interactions}}

\hypertarget{discussion}{%
\subsection{Discussion}\label{discussion}}

\hypertarget{limitations}{%
\subsection{Limitations}\label{limitations}}

\hypertarget{future-implications}{%
\subsection{Future Implications}\label{future-implications}}

\end{document}
